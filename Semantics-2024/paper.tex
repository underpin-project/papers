\documentclass[manuscript,screen]{acmart}


\IfFileExists{upquote.sty}{\usepackage{upquote}}{}
\IfFileExists{microtype.sty}{% use microtype if available
  \usepackage[]{microtype}
  \UseMicrotypeSet[protrusion]{basicmath} % disable protrusion for tt fonts
}{}
\makeatletter
\@ifundefined{KOMAClassName}{% if non-KOMA class
  \IfFileExists{parskip.sty}{%
    \usepackage{parskip}
  }{% else
    \setlength{\parindent}{0pt}
    \setlength{\parskip}{6pt plus 2pt minus 1pt}}
}{% if KOMA class
  \KOMAoptions{parskip=half}}
\makeatother

%%
%% This is file `sample-manuscript.tex',
%% generated with the docstrip utility.
%%
%% The original source files were:
%%
%% samples.dtx  (with options: `manuscript')
%% 
%% IMPORTANT NOTICE:
%% 
%% For the copyright see the source file.
%% 
%% Any modified versions of this file must be renamed
%% with new filenames distinct from sample-manuscript.tex.
%% 
%% For distribution of the original source see the terms
%% for copying and modification in the file samples.dtx.
%% 
%% This generated file may be distributed as long as the
%% original source files, as listed above, are part of the
%% same distribution. (The sources need not necessarily be
%% in the same archive or directory.)
%%
%%
%% Commands for TeXCount
%TC:macro \cite [option:text,text]
%TC:macro \citep [option:text,text]
%TC:macro \citet [option:text,text]
%TC:envir table 0 1
%TC:envir table* 0 1
%TC:envir tabular [ignore] word
%TC:envir displaymath 0 word
%TC:envir math 0 word
%TC:envir comment 0 0
%%
%%
%% The first command in your LaTeX source must be the \documentclass command.


% Options for packages loaded elsewhere
\PassOptionsToPackage{unicode}{hyperref}
\PassOptionsToPackage{hyphens}{url}
\PassOptionsToPackage{dvipsnames,svgnames,x11names}{xcolor}

\IfFileExists{bookmark.sty}{\usepackage{bookmark}}{\usepackage{hyperref}}

%% PANDOC PREAMBLE BEGINS


\providecommand{\tightlist}{%
  \setlength{\itemsep}{0pt}\setlength{\parskip}{0pt}}\usepackage{longtable,booktabs,array}
\usepackage{calc} % for calculating minipage widths
% Correct order of tables after \paragraph or \subparagraph
\usepackage{etoolbox}
\makeatletter
\patchcmd\longtable{\par}{\if@noskipsec\mbox{}\fi\par}{}{}
\makeatother
% Allow footnotes in longtable head/foot
\IfFileExists{footnotehyper.sty}{\usepackage{footnotehyper}}{\usepackage{footnote}}
\makesavenoteenv{longtable}
\usepackage{graphicx}
\makeatletter
\def\maxwidth{\ifdim\Gin@nat@width>\linewidth\linewidth\else\Gin@nat@width\fi}
\def\maxheight{\ifdim\Gin@nat@height>\textheight\textheight\else\Gin@nat@height\fi}
\makeatother
% Scale images if necessary, so that they will not overflow the page
% margins by default, and it is still possible to overwrite the defaults
% using explicit options in \includegraphics[width, height, ...]{}
\setkeys{Gin}{width=\maxwidth,height=\maxheight,keepaspectratio}
% Set default figure placement to htbp
\makeatletter
\def\fps@figure{htbp}
\makeatother

\definecolor{mypink}{RGB}{219, 48, 122}
\makeatletter
\@ifpackageloaded{caption}{}{\usepackage{caption}}
\AtBeginDocument{%
\ifdefined\contentsname
  \renewcommand*\contentsname{Table of contents}
\else
  \newcommand\contentsname{Table of contents}
\fi
\ifdefined\listfigurename
  \renewcommand*\listfigurename{List of Figures}
\else
  \newcommand\listfigurename{List of Figures}
\fi
\ifdefined\listtablename
  \renewcommand*\listtablename{List of Tables}
\else
  \newcommand\listtablename{List of Tables}
\fi
\ifdefined\figurename
  \renewcommand*\figurename{Figure}
\else
  \newcommand\figurename{Figure}
\fi
\ifdefined\tablename
  \renewcommand*\tablename{Table}
\else
  \newcommand\tablename{Table}
\fi
}
\@ifpackageloaded{float}{}{\usepackage{float}}
\floatstyle{ruled}
\@ifundefined{c@chapter}{\newfloat{codelisting}{h}{lop}}{\newfloat{codelisting}{h}{lop}[chapter]}
\floatname{codelisting}{Listing}
\newcommand*\listoflistings{\listof{codelisting}{List of Listings}}
\makeatother
\makeatletter
\makeatother
\makeatletter
\@ifpackageloaded{caption}{}{\usepackage{caption}}
\@ifpackageloaded{subcaption}{}{\usepackage{subcaption}}
\makeatother
%% PANDOC PREAMBLE ENDS

\setlength{\parindent}{10pt}
\setlength{\parskip}{0pt}

\hypersetup{
  pdftitle={Consolidating Wind Farm Data in Data Spaces for Training Predictive Maintenance ML},
  pdfauthor={Robert David; Axel Weißenfeld; Petar Ivanov; Nikola Tulechki; Vladimir Alexiev},
  colorlinks=true,
  linkcolor={blue},
  filecolor={Maroon},
  citecolor={Blue},
  urlcolor={red},
  pdfcreator={LaTeX via pandoc, via quarto}}

%% \BibTeX command to typeset BibTeX logo in the docs
\AtBeginDocument{%
  \providecommand\BibTeX{{%
    Bib\TeX}}}

%% Rights management information.  This information is sent to you
%% when you complete the rights form.  These commands have SAMPLE
%% values in them; it is your responsibility as an author to replace
%% the commands and values with those provided to you when you
%% complete the rights form.
\setcopyright{acmcopyright}
\copyrightyear{}
\acmYear{}
\acmDOI{}

%% These commands are for a PROCEEDINGS abstract or paper.
\acmConference[]{}{}{}
\acmPrice{}
\acmISBN{}

%% Submission ID.
%% Use this when submitting an article to a sponsored event. You'll
%% receive a unique submission ID from the organizers
%% of the event, and this ID should be used as the parameter to this command.
%%\acmSubmissionID{123-A56-BU3}

%%
%% For managing citations, it is recommended to use bibliography
%% files in BibTeX format.
%%
%% You can then either use BibTeX with the ACM-Reference-Format style,
%% or BibLaTeX with the acmnumeric or acmauthoryear sytles, that include
%% support for advanced citation of software artefact from the
%% biblatex-software package, also separately available on CTAN.
%%
%% Look at the sample-*-biblatex.tex files for templates showcasing
%% the biblatex styles.
%%

%%
%% The majority of ACM publications use numbered citations and
%% references.  The command \citestyle{authoryear} switches to the
%% "author year" style.
%%
%% If you are preparing content for an event
%% sponsored by ACM SIGGRAPH, you must use the "author year" style of
%% citations and references.
%% Uncommenting
%% the next command will enable that style.
%%\citestyle{acmauthoryear}


%% end of the preamble, start of the body of the document source.
\begin{document}


%%
%% The "title" command has an optional parameter,
%% allowing the author to define a "short title" to be used in page headers.
\title{Consolidating Wind Farm Data in Data Spaces for Training
Predictive Maintenance ML}

%%
%% The "author" command and its associated commands are used to define
%% the authors and their affiliations.
%% Of note is the shared affiliation of the first two authors, and the
%% "authornote" and "authornotemark" commands
%% used to denote shared contribution to the research.


  \author{Robert David}
  
            \affiliation{%
                  \institution{The Semantic Web Company}
                                                  \country{Austria}
                      }
        \author{Axel Weißenfeld}
  
            \affiliation{%
                  \institution{Austrian Institute of Technology}
                                                  \country{Austria}
                      }
        \author{Petar Ivanov}
  
    \author{Nikola Tulechki}
  
    \author{Vladimir Alexiev}
  \orcid{0000-0001-7508-7428}
            \affiliation{%
                  \institution{Ontotext}
                                                  \country{Bulgaria}
                      }
      

%% By default, the full list of authors will be used in the page
%% headers. Often, this list is too long, and will overlap
%% other information printed in the page headers. This command allows
%% the author to define a more concise list
%% of authors' names for this purpose.
%\renewcommand{\shortauthors}{Trovato et al.}
%%  
%% The abstract is a short summary of the work to be presented in the
%% article.
\begin{abstract}
The UNDERPIN project addresses the challenge of training machine
learning (ML) models for predictive maintenance of wind farms by
leveraging consolidated training data provided by multiple partners.
This collaborative effort is facilitated through a European-wide Data
Space for Manufacturing for dynamic asset management and predictive and
prescriptive maintenance established as part of the project. Data spaces
allow trusted participants to share data while maintaining data
sovereignty. However, consolidating diverse data sets requires a clear
semantics and mapping framework. To solve this, the project employs
Semantic Web standards and technologies to model and process the data.
Using OWL (Web Ontology Language), vocabularies and ontologies are
developed to represent the data with clear, standardized semantics. Two
key software components, GraphDB and PoolParty, are utilized to
implement these standards and provide a semantic layer to efficiently
process the consolidated data. The main benefit is to support ML
training by presenting a consolidated and uniform training data set. Our
approach aims to enhance the quality and reliability of predictive
maintenance models for wind farms. A demonstration from the UNDERPIN
project showcases the process of modeling and processing the data,
illustrating the practical application of these semantic technologies.
This example highlights the project's approach and the potential
improvements in predictive maintenance through enhanced data integration
and processing.    
\end{abstract}

%%
%% The code below is generated by the tool at http://dl.acm.org/ccs.cfm.
%% Please copy and paste the code instead of the example below.
%%

%%
%% Keywords. The author(s) should pick words that accurately describe
%% the work being presented. Separate the keywords with commas.
\keywords{data spaces, oil and gas, renewable
energy, refineries, windfarms, time series, ontologies, semantic
technologies}


%%
%% This command processes the author and affiliation and title
%% information and builds the first part of the formatted document.
\maketitle

\setlength{\parskip}{-0.1pt}

\section{Introduction}\label{introduction}

Consolidating wind farm data in data spaces for training predictive
maintenance ML - Use Case Description

\section{Initial Situation}\label{initial-situation}

In the UNDERPIN project, we face the challenge of training ML for
predictive maintenance of wind farms. This scenario benefits from large
amounts of training data. Such data can be provided by different
partners who collaborate on this ML approach. We support this scenario
by implementing a data space for wind farm data. Data spaces enable data
sharing between trusted participants while keeping data sovereignty.
However, for consolidating different data sets from participants, we
need to establish a clear semantics and mapping for the data.

\section{Approach and IT-Solution}\label{approach-and-it-solution}

To tackle this problem, we introduce Semantic Web standards and
technologies to model and process this data. Vocabularies and
ontologies, specifically expressed using OWL, can represent such data
with a clear semantics based on standards. Furthermore, we introduce
GraphDB and PoolParty as two software components which implement these
standards to process the data.

The Wind Farm Ontology \citep{ikeresnaola-gonzalezWindFarmOntology2021}
\ldots{}

\section{Business Value and Benefits of the Semantic
Solution}\label{business-value-and-benefits-of-the-semantic-solution}

When consolidating the data, we need to provide a method to
automatically map and process it so this can be done efficiently on
large data sets and/or regular updates in data spaces. The benefit
provided should support the ML training in such a way that the
consolidated data set is exposed as uniform training data set.

\section{Prospects and
Recommendations}\label{prospects-and-recommendations}

We provide a first prototype to implement our approach. Next steps are
bringing it into the actual data space in practice and representing and
integrating further data sets from different data space participants.
Also, the ML should actually benefit regarding quality in practice,
which still needs to be verified.

\section{Conclusion}\label{conclusion}

\bibliographystyle{ACM-Reference-Format}
\bibliography{bibliography.bib}

%% begin pandoc before-bib
%% end pandoc before-bib
%% begin pandoc biblio
%% end pandoc biblio
%% begin pandoc include-after
%% end pandoc include-after
%% begin pandoc after-body
%% end pandoc after-body

\end{document}
\endinput
%%
%% End of file `sample-manuscript.tex'.
